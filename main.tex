\documentclass{article}
\usepackage{graphicx} % Required for inserting images
\usepackage[rightcaption]{sidecap}
\usepackage[style=alphabetic]{biblatex}

\graphicspath{ {./images/} }
\addbibresource{sample.bib}

\title{Assignment 4 Report}
\author{Ole Rößler (7211)}
\date{December 2024}

\begin{document}

\maketitle
\tableofcontents

\newpage
\section{Design Exercise: Flight Booking System}

\subsection{Requirements}
This section talks about the functional and non-functional requirements of the flight booking system wanted by the ACME Organization. 
The main points given were about the given points considering (this will be a short overview about the task).

Short Description what the differences between function and non-function requirements are and I chose them (from the given context and my personal thought).

\subsubsection{Functional Requirements}
\begin{enumerate}

\item \textbf{Fetching Data:}
Periodically receive flight data updates from partner airlines (every hour).
\item \textbf{View Flights:}
The customer is able to see specific flight details. 
Therefore the customer enters a specific flight number/ flight ID and gets updates about the specified flight.
\item \textbf{Booking:}
The customer is able to book a flight on a desired flight as well as make a seat reservation on the airplane (Most airlines offer different price-tiers for seats). 
\end{enumerate}

\subsubsection{Non-Functional Requirements}
\begin{enumerate}
\item \textbf{Price Consistency:}
The system has to maintain consistency in pricing, therefore short term increases in flight-prices have to be avoided .
\item \textbf{Performance:}
Low latency for user interactions (response within seconds).
\item \textbf{Scalability:}
The system has to handle hundreds of thousands of concurrent users as well as hundreds of flights send in their data on the hour.
\item \textbf{Fault tolerance:}
The system has to be reliable.
\end{enumerate}

\subsection{System Architecture}
\begin{figure}
    \centering
    \includegraphics[width=\linewidth]{1st_system_design_sketch.png}
    \caption{Redesigned flight booking system (FBS)}
    \label{fig:enter-label}
\end{figure}
\newpage 

\subsection{Discussion}
The functional requirements are mainly implemented as micro-services.
The Flight-Search service queries the database according to the users flight-data. The booking service is used to book a flight that 

The Seat-Select service is called if a customer wants to make a seat reservation on a selected plane. 
The proxy handles...security before forwarding to service, also record performance metrics as entry-point???

In order to address the non-functional requirements and insure scalability and performance different measures were taken. 
The Load-Balancer distributes the incoming traffic equally to the proxy-servers so that the performance of the system isn't impacted by a single overloaded server (that would take way to long to respond to the user-requests). This also enables another point of scalability such that we can increase the number of proxy servers if the overall workload is to high.
My analysis of the given flight-booking-system was, that it has a high write-to-read ratio (meaning the number of writes is way higher than the number of reads). Therefore we have database dublicates that all follow a single leader. With the additional caching it is insured, that the database-reads don't bottleneck the application. In the future additional replica can be added to support the database layer. We could also add additional partitioning (by origin-destination). 

To scale the micro-services it is always possible to horizontally scale the micro-services by adding more instances. If the request arrives at the service it will be load-balanced internally and send to one of its instances.


\section{Freestyle Exercise}
\subsection{My DS-Concept}
I chose to implement a distributed version of our assingment server. 
that is able to detect nodes that fail and automatically replaces them.
->read/write to database 
->basic flight protocol, i can receive flight information and write to the database using ???

\subsection{Evaluation}
The evaluation is based on...


-Problems, what happens to failed nodes, can we restart them or something (do they just dangle on a server)?

-What happends if the leader/manager of the system fails:
-Implement distributed 
\printbibliography[heading=bibintoc]
\end{document}
